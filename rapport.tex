\documentclass{article}
\usepackage[T1]{fontenc} 
\usepackage[utf8]{inputenc}
\usepackage[french]{babel}
\usepackage{listings, xcolor, graphicx}
\usepackage{amsmath}
\graphicspath{ {./} }

\lstset{frame=tb,
   language=java,
   aboveskip=3mm,
   belowskip=3mm,
   showstringspaces=false,
   columns=flexible,
   basicstyle={\small\ttfamily},
   numbers=none,
   numberstyle=\tiny\color{gray},
   keywordstyle=\color{blue},
   commentstyle=\color{dkgreen},
   stringstyle=\color{mauve},
   breaklines=true,
   breakatwhitespace=true
   tabsize=3
}


\title{Modélisation de la propagation des variants du COVID}
\author{Bouzakri Wassim, Coupet Joss, Marchetti Marie-Eden, Vasseur Pierre-Adrien}
\date{Février 2022}

\begin{document}

\maketitle


\section{Résumé}

Dans ce rapport, ...

\section{Introduction}

Dans ce rapport, on va étudier la propagation de la Covid-19 par le biais des variants.\\
On se demande alors si leur circulation et mutation influencent le comportement de l'épidémie, en plus de considérer leur interaction entre eux.\\



Problématique :
- La circulation et la mutation de variants influencent-elles le comportement de l'épidémie de la Covid-19 ?\\
- Comment les variants intéragissent-ils entre eux ?

Commentaire sur l'existant :\\
Même chose existe et déjà fait?
Justifier ce que l'on a produit en se référençant à ce qui existe?
Citer les sources
Si non spécialiste et veut savoir comment ça marche, n'importe qui doit pouvoir lire, comprendre, reproduire le rapport
- Comment on peut d'une manière simple modéliser la propagation de la Covid-19, manière d'apparenter la chose

\section{Modélisation de départ}

Les variables de base pour seulement 1 virus : 
\begin{align}
    S(t)= \text{\% de personnes saines dans la population au temps t} \\
    I(t)= \text{\% de personnes contaminées dans la population au temps t} \\
    R(t)= \text{\% de personnes en rémission dans la population au temps t} \\
\end{align}

Mise en équation et système : 
\begin{align}
    \dot{S}(t)= -\alpha S(t)I(t) \\
    \dot{I}(t)= \alpha S(t)I(t)-\beta I(t) \\
    \dot{R}(t)= \beta I(t) + \beta I(t)
\end{align}

\section{Modélisation avancée}


Exemple avec 2 variants.

Les variables : 
\begin{align}
    S_1(t)= \text{\% de non infecté par le variant 1.} \\
    S_2(t)= \text{\% de non infecté par le variant 2.} \\
    I_1(t)= \text{\% d'infecté au variant 1.} \\
    I_2(t)= \text{\% d'infecté au variant 2.} \\
    R_1(t)= \text{\% de remis au variant 1.} \\
    R_2(t)= \text{\% de remis au variant 2.} \\
\end{align}

Le système d'équation pour le premier variant : 
\begin{align}
    \dot{S_1}(t)= -\alpha_1 S_1(t)I_1(t) \\
    \dot{I_1}(t)= \alpha_1 S_1(t)I_1(t)-\beta_1 I_1(t) \\
    \dot{R_1}(t)= \beta_1 I_1(t)
\end{align}
On obtient le même avec le second variant. \\
Cependant le système est trop simple, il n'y a pas d'interaction entre les virus. \\
On fait un nouveau système qui permet une interaction entre les variants :
\begin{align}
    \dot{S}(t)= -\alpha_1 S(t)I_1(t) - \alpha_2 S(t)I_2(t) \\
    \dot{I_1}(t)= \alpha_1 S(t)I_1(t)-\beta_1 I_1(t) \\
    \dot{I_2}(t)= \alpha_2 S(t)I_2(t)-\beta_2 I_2(t) \\
    \dot{R}(t)= \beta_1 I_1(t) + \beta_2 I_2(t)
\end{align}
On part du principe que l'on ne peut être contaminé que par un variant. \\
On veut maintenant modéliser les mutations.\\
Le \% de chance de mutation du variant 1 sur l'intervalle de temps T est défini par :
\begin{align}
    mut = I_1(t)*(1-e^{\gamma T})\text{ avec }\gamma > \text{0}
\end{align}

\section{Modèle de l'étude}

\section{Simulation numérique}

\section{Interprétation des résultats}

\section{Conclusion}

reprendre la problématique, réponse
améliorations? perspectives?

\section{Référence}

Modeling neutral viral mutations in the spread of SARS-CoV-2 epidemics -> 
https://journals.plos.org/plosone/article?id=10.1371/journal.pone.0255438

Modeling neutral viral mutations in the spread ofSARS-CoV-2 epidemics ->
https://figshare.com/articles/journal_contribution/Simulation_parameters_analytical_calculations_real_genetic_evolution_algorithm_and_Chinese_epidemic_data_corrections_/15076462




\end{document}
