\documentclass{article}
\usepackage[T1]{fontenc} 
\usepackage[utf8]{inputenc}
\usepackage[french]{babel}
\usepackage{listings, xcolor, graphicx}
\usepackage{amsmath}
\graphicspath{ {./} }

\lstset{
    tabsize = 4, %% set tab space width
    showstringspaces = false, %% prevent space marking in strings, string is defined as the text that is generally printed directly to the console
    numbers = left, %% display line numbers on the left
    commentstyle = \color{green}, %% set comment color
    keywordstyle = \color{blue}, %% set keyword color
    stringstyle = \color{red}, %% set string color
    rulecolor = \color{black}, %% set frame color to avoid being affected by text color
    basicstyle = \small \ttfamily , %% set listing font and size
    breaklines = true, %% enable line breaking
    numberstyle = \tiny,
}

\title{Modélisation de la propagation des variants du COVID}
\author{Bouzakri Wassim, Coupet Joss, Marchetti Marie-Eden, Vasseur Pierre-Adrien}
\date{Decembre 2021}

\begin{document}

\maketitle

\section{Le modèle de base}

Les variables de base pour seulement 1 virus : 
\begin{align}
    S(t)= \text{\% de personnes saines dans la population au temps t} \\
    I(t)= \text{\% de personnes contaminées dans la population au temps t} \\
    R(t)= \text{\% de personnes en rémission dans la population au temps t} \\
\end{align}

Mise en équation et système : 
\begin{align}
    \dot{S}(t)= -\alpha S(t)I(t) \\
    \dot{I}(t)= \alpha S(t)I(t)-\beta I(t) \\
    \dot{R}(t)= \beta I(t) + \beta I(t)
\end{align}

\section{Le modèle}

Exemple avec 2 variants.

Les variables : 
\begin{align}
    S_1(t)= \text{\% de non infecté par le variant 1.} \\
    S_2(t)= \text{\% de non infecté par le variant 2.} \\
    I_1(t)= \text{\% d'infecté au variant 1.} \\
    I_2(t)= \text{\% d'infecté au variant 2.} \\
    R_1(t)= \text{\% de remis au variant 1.} \\
    R_2(t)= \text{\% de remis au variant 2.} \\
\end{align}

Le système d'équation pour le premier variant : 
\begin{align}
    \dot{S_1}(t)= -\alpha_1 S_1(t)I_1(t) \\
    \dot{I_1}(t)= \alpha_1 S_1(t)I_1(t)-\beta_1 I_1(t) \\
    \dot{R_1}(t)= \beta_1 I_1(t)
\end{align}
On obtient le même avec le second variant. \\
Cependant le système est trop simple, il n'y a pas d'interaction entre les virus. \\
On fait un nouveau système qui permet une interaction entre les variants :
\begin{align}
    \dot{S}(t)= -\alpha_1 S(t)I_1(t) - \alpha_2 S(t)I_2(t) \\
    \dot{I_1}(t)= \alpha_1 S(t)I_1(t)-\beta_1 I_1(t) \\
    \dot{I_2}(t)= \alpha_2 S(t)I_2(t)-\beta_2 I_2(t) \\
    \dot{R}(t)= \beta_1 I_1(t) + \beta_2 I_2(t)
\end{align}
On part du principe que l'on ne peut être contaminé que par un variant. \\
On veut maintenant modéliser les mutations.\\
Le \% de chance de mutation du variant 1 sur l'intervalle de temps T est défini par :
\begin{align}
    mut = I_1(t)*(1-e^{\gamma T})\text{ avec }\gamma > \text{0}
\end{align}

\section{La simulation}

\end{document}
